\chapter{Conclusion}

Gatling offre une nouvelle approche du test de charge grâce à des performances optimisées, une façon d'écrire les scénarios plus naturelle et un langage résolument proche du vocabulaire utilisé dans le domaine du test de charge. Publié sous licence Apache 2, Gatling est donc Open Source et disponible à l'adresse http://github.com/excilys/gatling. La version actuelle pose des bases solides pour le développement de la suite des fonctionnalités prévues dans la roadmap, et j'espère qu'elle saura percer et devenir un outil incontournable pour le test de charge basé sur des scénarios d'utilisation.

Ce stage fut très intéressant et extrêmement enrichissant. La formation proposée m'a permis de découvrir de nouvelles technologies. La conception de Gatling m'a initié au domaine du test de charge que je ne connaissais pas, et m'a permis de participer à la communauté Open Source et de découvrir un nouveau langage de programmation. 

L'implication de l'entreprise dans le développement personnel de ses collaborateurs m'a permis de découvrir des communautés telles que le Paris Java User Group, l'association JDuchess et le Paris Android User Group ; je suis également devenu Oracle Certified Java Programmer. Ravi d'intégrer cette société, je pourrai continuer à m'investir dans le développement de Gatling sous la bannière Excilys :-)
