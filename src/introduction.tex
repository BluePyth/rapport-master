\chapter{Introduction}
\section{eBusiness Information et Excilys}

\subsection{La Charte Excilys}
eBusiness Information a été créée en l'an 2000 dans le but de répondre à des problématiques métier diverses grâce à la mise en place de solutions informatiques robustes, tout en mettant en valeur le savoir-faire humain qui amène à ces solutions.

En effet, pour ses fondateurs, le service informatique est un métier réalisé par des hommes qui doivent avoir de plus en plus de compétences. En 2002, ils imaginent une notion nouvelle dans le monde du service informatique : le \em{Service équitable}\cite{www_excilys}. Inspiré par le commerce équitable, cette notion sera inscrite dans une charte : la \em{Charte Excilys}. Le service équitable consiste en la création d'un cercle vertueux\footnote{L'entreprise facture le client plus cher que la moyenne. Cette facturation profite au consultant qui fournit donc un travail de qualité. Ce travail de qualité satisfait le client. Ainsi, les acteurs concernés par le service rendu y trouvent tous leur compte.} profitant à tous les acteurs du service. Tout comme le commerce équitable, cette notion vise à profiter aux personnes qui produisent le service plutôt qu'aux intermédiaires seuls.

La \em{Charte Excilys} définit donc un ensemble de droits et de devoirs entre les trois acteurs du service : le client, l'entreprise et le consultant comme montré figure \ref{droits_devoirs}.

\begin{figure}[h]
\begin{center}
\includegraphics[width=200pt]{img/droits_devoirs.jpg}
\end{center}
\caption{Droits et devoirs pour les acteurs du service équitable}
\label{droits_devoirs}
\end{figure}

L'application des règles de cette charte était un pari risqué, mais un pari réussi. En effet, après plusieurs années de fonctionnement, eBusiness Information a prouvé qu'un modèle basé sur les compétences humaines peut fonctionner. De plus, les résultats de l'entreprise sont extrêmement encourageants : relations durables avec les clients, aucunes dettes, aucun investisseur extérieur et aucun projet raté.

\subsection{Le groupe Excilys}
Le groupe Excilys est né de la volonté de reproduire ce modèle dans d'autres sociétés de service qui souhaitent elles aussi centrer leur métier sur l'excellence ; c'est d'ailleurs du mot excellence que vient le nom \em{Excilys}.

Depuis, plusieurs entreprises ont rejoint le groupe, elles sont aujourd'hui au nombre de sept : Adlys, Altendis, eBusiness Information, Edvance, Equitalis, SS2J et Visual3X. Chacune de ces sociétés fournit des services en informatique mais avec des domaines de compétences complémentaires et une image de qualité qui lui est propre.
  
\section{Formation Java/JEE}

\subsection{Être stagiaire chez eBusiness Information}

Excilys\footnote{Dans le reste de ce rapport, eBusiness Information et Excilys, bien que représentant deux entités juridiques différentes, désigneront l'entreprise où j'ai réalisé mon stage.} emploie de nombreux stagiaires chaque année ; par exemple, pendant l'année 2011, c'est plus d'une vingtaine de stagiaires qui a été accueillie chez Excilys. L'entreprise mise beaucoup sur les jeunes informaticiens en fin d'études afin d'embaucher des consultants compétents ; ce qui lui permettra de continuer à offrir un service de qualité.

Dans cette optique, un grand soin est apporté au confort des stagiaires : hébergement disponible, salaire confortable, et, surtout, une formation de six semaines au début du stage afin de leur permettre de monter en compétence sur de nombreuses technologies. Le fait que les stagiaires soient en colocation dans le même immeuble, et qu'ils travaillent dans le même open space permet de créer rapidement des liens et cela participe à une émulation générale, ainsi qu'à une bonne ambiance.

A la suite de la formation, chaque stagiaire se voit proposer un sujet de stage différent, et parfois, une mission chez un client qui se poursuit fréquemment après la fin du stage. 

\subsection{Capico}

Afin de former leurs stagiaires et d'offrir à leurs consultants la possibilité de se former facilement sur diverses technologies utilisées dans le domaine Java/JEE, les gérants d'eBusiness Information ont investi dans le développement d'un outil de formation en ligne nommé \em{Capico}\footnote{Capico pour \em{Capi}talisation des \em{co}nnaissances}. Cet outil contient des cours (audio ou non) ainsi que des exercices et travaux pratiques avec leurs corrigés, la figure \ref{capico} représente l'interface (côté élève) de Capico.

\begin{figure}[ht]
\begin{center}
\includegraphics[width=400pt]{img/capico.png}
\end{center}
\caption{Interface de Capico}
\label{capico}
\end{figure}

Aujourd'hui, Capico est devenu bien plus qu'un outil de formation interne. En effet, avec le temps, et grâce aux investissements en R\&D d'Excilys, ce logiciel est devenu une plateforme d'e-coaching qui propose des cours gratuits du CP au CM2, en français et mathématiques\cite{capicofr} et qui est récemment entré en compétition dans un appel d'offres du département des Hauts-de-Seine afin de moderniser ses collèges.  

\subsection{Contenu de la formation}

\subsubsection{Cours sur Capico}
La période de six semaines de formation commence par des cours sur Capico couvrant diverses technologies mais aussi des notions sur les méthodes agiles, en particulier \em{eXtreme Programming}. Les technologies abordées sont les suivantes :

\begin{itemize}
  \item \em{UML / Java 5} - Un rappel sur Java et le langage de formalisation UML ;
  \item \em{JEE JSP+Servlets / Java EE 6} - Un rappel sur Java Enterprise (EJB, Servlet, JSP, JSF) ;
  \item \em{JDBC / Hibernate} - Un cours sur JDBC et Hibernate, un des ORM Java les plus utilisés en entreprise ;
  \item \em{Log4j / Slf4j} - Un cours sur la journalisation en Java ;
  \item \em{JUnit} - Un cours sur le framework JUnit qui permet de faire du test unitaire ;
  \item \em{Spring} - Un cours sur le framework Spring (IoC, MVC, etc.) ;
  \item \em{Maven} - Un cours sur un gestionnaire de dépendances de plus en plus utilisé au sein de la communauté Java ;
  \item \em{Subversion / Git} - Un cours sur les deux gestionnaires de sources les plus populaires ;
  \item \em{Flex} - Un cours expliquant les bases du développement avec Flex.
\end{itemize}

Ces cours offrent un tour d'horizon des technologies les plus utilisées par les clients de l'entreprise, ainsi que les technologies utilisées au sein du projet Capico. En effet, chaque stagiaire travaille sur le projet Capico avant d'être affecté à une mission ou à un projet différent, dans le but de mettre en pratique ce qu'il a appris sur un projet concret.

\subsubsection{Réalisation d'un projet}

La formation, est complétée par la mise en pratique des cours grâce à la réalisation d'une application d'e-banking dans le cadre des méthodes agiles. Notre groupe de six stagiaires à réalisé son application\footnote{Disponible sur Github : http://github.com/BluePyth/patricks-bank} sur cinq semaines, soit cinq itérations en développant par binôme tel que préconisé par la méthode eXtreme Programming. La figure \ref{patricks_bank} représente une page du site que nous avons réalisé.

\begin{figure}[h]
\begin{center}
\includegraphics[width=400pt]{img/patricks_bank.png}
\end{center}
\caption{Patrick's Bank - Visualisation du détail d'un compte}
\label{patricks_bank}
\end{figure}

Nous avons pu ainsi nous confronter aux technologies que nous n'avions jusqu'alors pas pratiquées : Intégration Continue avec Jenkins, utilisation de Google code, de Maven, etc.

\section{Ma mission}
A la fin de ma période de formation, j'ai commencé par travailler sur Capico pendant deux semaines. Puis Stéphane Landelle, directeur technique d'eBusiness Information, m'a proposé de travailler avec lui sur une idée dont il avait déjà réalisé une \em{Preuve de Concept}. Cette idée consiste en la création d'un nouvel outil de test de montée en charge $-$ ou injecteur HTTP $-$ nommé Gatling\cite{www_gatling}.

Dès lors, j'ai été en charge de concevoir, puis d'implémenter les fonctionnalités que l'outil doit proposer. L'objet de ce rapport consiste en la présentation de la réalisation de ce nouvel outil.

Le chapitre \ref{chap_test_charge} définira ce qu'est un test de charge et motivera la création d'un nouvel outil permettant d'automatiser ce type de tests. Le chapitre \ref{chap_pb} décrira l'analyse des fonctionnalités que Gatling devra implémenter afin de trouver des solutions aux divers problèmes soulevés par la création d'un tel outil. Enfin, le chapitre \ref{chap_conception} détaillera la conception de Gatling et listera la plupart des fonctionnalités disponibles dans sa version actuelle, la 1.0.0.M3.