\chapter{Glossaire}
\begin{description}
\item[Denial of Service] Attaque informatique visant à rendre inopérant un serveur en lui envoyant un très grand nombre de requêtes par seconde.
\item[CPU] \en{Central Processing Unit}. La CPU, ou microprocesseur, est une puce informatique capable d'effectuer des calculs. C'est le ``cerveau'' de l'ordinateur.
\item[AJAX] \en{Asynchronous Javascript And XML}. AJAX est le regroupement sous un seul nom, de différentes technologies permettant de créer des interfaces web plus riches et d'offrir une expérience utilisateur plus agréable.
\item[HTTP] \en{HyperText Transfer Protocol}. HTTP est le protocole utilisé par les navigateurs internet et les serveurs web afin de communiquer.
\item[LDAP] \en{Lightweight Directory Access Protocol}. LDAP est une norme pour les annuaires informatiques.
\item[DSL] \en{Domain Specific Language}. Un DSL ou Langage spécifique à un domaine, permet de décrire des entités grâce à un langage créé uniquement dans ce but.
\item[IDE] \en{Integrated Development Environment}. Un environnement de développement intégré est un logiciel qui permet au développeur de créer facilement des applications.
\item[PoC] \en{Proof of Concept}. Une preuve de concept est une application très simple visant à valider un concept avant de l'utiliser à plus grande échelle.
\item[JVM] \en{Java Virtual Machine}. La JVM est une application qui permet l'exécution de programmes compilés en bytecode, le plus souvent écrits en Java. 
\item[REST et RESTful] \en{REpresentational State Transfer}. Architecture d'applications fonctionnant autour de ressources et utilisant les verbes HTTP afin d'agir sur ces ressources.
\end{description} 